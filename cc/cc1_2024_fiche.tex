\documentclass[landscape, utf8, 10pt]{article}
% \documentclass[utf8, 10pt]{fiche_xls}

\usepackage{fiche}
\usepackage[landscape]{geometry}

\begin{document}
\begin{multicols}{2}

\nom{21.5}{20}

\bigskip

\exo{1}{Limites planétaires}{1}

\sol{1}{1}{aggravation du réchauffement climatique $\rightarrow$ diminution de la biodiversité (et réciproquement... par ex.)}{1} \\


\exo{2}{Energie potentielle des marrées}{12.5}

\sol{2}{2}{convertit l'énergie mécanique en énergie électrique}{0.5} \\
\sol{2}{3}{$ E_{pp} = \displaystyle \frac{\rho_e Sh^2g}{2}$}{1} \\
\sol{2}{4}{même efficacité qu'un barrage hydroélectrique~: environ 90 $\%$}{1} \\
\sol{2}{5}{$P_{max} = \frac{4\eta E_{pp}}{2T} = \frac{\eta\rho_e Sh^2g}{T}$}{1} \\
\sol{2}{6}{$P_{max}$ = 295 MW}{1} \\
\sol{2}{6}{très supérieure à une éolienne (de l'ordre de 5 MW) mais inférieure à un réacteur nucléaire (de l'ordre de 1 GW)}{0.5} \\
\sol{2}{7}{$ P_{PV} = \eta_{PV}E = 660$ MW et donc $P_{max}/P_{PV}$ = 0,45}{1} \\
\sol{2}{7}{puissance produite par m$^2$ au sol est supérieure dans le cas du photovoltaïque}{0.5} \\
\sol{2}{8a}{$D_v = \frac{\Delta V}{\tau} = \frac{hS\cos(\omega\tau/2)}{\tau}$}{1} \\
\sol{2}{8b}{$ P_e = \eta \rho_e g D_v (z_2-z_1) $}{1} \\
\sol{2}{8c}{$ E_e = \int_{t = (T-\tau)/2}^{(T+\tau)/2}P_e(t) dt = \eta \rho_e g h^2S \frac{\sin(\omega\tau/2)\cos(\omega\tau/2)}{\omega\tau} = \eta\frac{\rho_e g h^2S}{2}\times\frac{\sin(\omega\tau)}{\omega\tau}$}{2.5} \\
\sol{2}{8d}{$\eta' = \eta \frac{\sin(\omega\tau)}{\omega\tau}$}{0.5} \\
\sol{2}{8d}{graphe correct}{1} \\


\exo{3}{Énergie cinétique des courants marins}{4.5}

\sol{3}{9}{$[P_{cin}] = [\rho_e S v^3] = M.L^2.T^{-3}$}{0.5} \\
\sol{3}{10}{$P_{cin} =$ 246 GW}{1} \\
\sol{3}{10}{consommation mondiale $\sim$ 20 TW (en énergie primaire); $P_{cin}$ représente $\sim$ 1 \% de cette consommation}{0.5} \\
\sol{3}{11}{$C_P = \frac{P}{P_{cin}}$}{0.5} \\
\sol{3}{11}{$C_P$ ne peut pas dépasser la limite de Betz}{0.5} \\
\sol{3}{12}{Puissance traversée par l'éolienne : $\frac{(2\times \pi R^2)\rho_ev^3}{2} = 2,8 \ \text{MW}$i, soit $C_P = 1,2/2.8 = 0,43$}{1.5} \\


\exo{4}{Consommation d'énergie et effet de serre}{3.5}

\sol{4}{13}{$\mathcal P_h = \displaystyle \frac{6,72.10^{20}}{(3,14.10^7)}$ = 2,1.10$^{13}$ W, soit 21 TW}{1} \\
\sol{4}{13}{80 \% de cette énergie est fossile, et 3-4 \% est nucléaire}{0.5} \\
\sol{4}{14}{$\Delta \mathcal P_{GES} = \Delta p \times S_T = 2,3 \times 510.10^{12}$ = 1,17.10$^{15}$ W = 1170 TW}{1} \\
\sol{4}{15}{$\displaystyle \frac{\mathcal P_{GES}} {\mathcal P_h} = 1170/21 \sim 56$}{0.5} \\
\sol{4}{15}{$E_h = \displaystyle \frac{6,7.10^{20}}{10^{14}} \sim$ 6 millions de bombes annuelles et $E_{GES} \sim$ 340 millions de bombes annuelles}{0.5} \\


\end{multicols}

\end{document}
